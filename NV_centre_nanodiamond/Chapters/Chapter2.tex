\chapter{Power absorption for beyond Rayleigh size particles: Integral equation form of Mie scattering}

\section{Setup}
We consider a time-harmonic electromagnetic wave, with time dependence $e^{-i\omega t}$, incident on a scattering particle.
\begin{itemize}
    \item The exterior medium is homogeneous and lossless, with real permittivity $\varepsilon_m$ and permeability $\mu_0$.
    \item The total electric and magnetic fields outside the particle are a superposition of the incident and scattered fields: $\vv{E} = \vv{E}_{\text{inc}} + \vv{E}_{\text{sca}}$ and $\vv{H} = \vv{H}_{\text{inc}} + \vv{H}_{\text{sca}}$.
    \item The time-averaged Poynting vector is defined as $\vv{S} := \frac{1}{2} \Re(\vv{E} \times \vv{H}^*)$.
\end{itemize}
Our goal is to prove the relationship between absorbed, scattered, and extinction powers.

\section{Derivation of the Optical Theorem Power Balance: \\ \large{$P_{\text{abs}} = P_{\text{ext}} - P_{\text{sca}}$}}

\subsection{Time Averaged Poynting Theorem}
The time-averaged Poynting theorem states that the net power flowing out of a closed surface is equal to the negative of the power dissipated (absorbed) within the volume enclosed by that surface. Let's consider a large sphere $S_R$ of radius $R$ that encloses the scattering particle. The net power flux out of this sphere is (derived in Appendix B Eq. \eqref{B_P_abs}):
\begin{equation}
    \oint_{S_R} \avg{\vv{S}} \cdot \hat{\vv{r}} \,dA = -P_{\text{abs}}
    \label{eq:poynting}
\end{equation}
Here, $P_{\text{abs}} = \frac{\omega}{2} \int_{\text{particle}} \varepsilon'' |\vv{E}|^2 \,dV$ is the total power absorbed by the particle. The left-hand side represents the total power leaving the sphere; if the particle absorbs energy, this net flux must be negative (i.e., more power flows in than out).

\subsection{Decomposing into the Power Flux}
We now express the total Poynting vector $\avg{\vv{S}}$ in terms of the incident and scattered fields.
\begin{align}
    \vv{E} \times \vv{H}^* &= (\vv{E}_{\text{inc}} + \vv{E}_{\text{sca}}) \times (\vv{H}_{\text{inc}} + \vv{H}_{\text{sca}})^* \\
    &= (\vv{E}_{\text{inc}} + \vv{E}_{\text{sca}}) \times (\vv{H}_{\text{inc}}^* + \vv{H}_{\text{sca}}^*) \\
    &= \underbrace{\vv{E}_{\text{inc}} \times \vv{H}_{\text{inc}}^*}_{\text{Incident}} + \underbrace{\vv{E}_{\text{sca}} \times \vv{H}_{\text{sca}}^*}_{\text{Scattered}} + \underbrace{\vv{E}_{\text{inc}} \times \vv{H}_{\text{sca}}^* + \vv{E}_{\text{sca}} \times \vv{H}_{\text{inc}}^*}_{\text{Interference (Extinction)}}
\end{align}
By taking $\frac{1}{2}\Re\{\cdot\}$ of each term, we can split the total Poynting vector $\avg{\vv{S}}$ into three meaningful components:
\begin{align}
    \avg{\vv{S}_{\text{inc}}} &:= \frac{1}{2} \Re(\vv{E}_{\text{inc}} \times \vv{H}_{\text{inc}}^*) \\
    \avg{\vv{S}_{\text{sca}}} &:= \frac{1}{2} \Re(\vv{E}_{\text{sca}} \times \vv{H}_{\text{sca}}^*) \\
    \avg{\vv{S}_{\text{int}}} &:= \frac{1}{2} \Re(\vv{E}_{\text{inc}} \times \vv{H}_{\text{sca}}^* + \vv{E}_{\text{sca}} \times \vv{H}_{\text{inc}}^*)
\end{align}
The total flux is the sum of the fluxes from these three parts:
\begin{equation}
    \oint_{S_R} \avg{\vv{S}} \cdot \hat{\vv{r}} \,dA = \oint_{S_R} \avg{\vv{S}_{\text{inc}}} \cdot \hat{\vv{r}} \,dA + \oint_{S_R} \avg{\vv{S}_{\text{sca}}} \cdot \hat{\vv{r}} \,dA + \oint_{S_R} \avg{\vv{S}_{\text{int}}} \cdot \hat{\vv{r}} \,dA
    \label{2_eq:split}
\end{equation}

\subsection{Analysing Each Flux Integral}
We now evaluate each of the three integrals on the right-hand side of Eq.~\eqref{2_eq:split}.

\paragraph{Incident Flux:} The incident fields $(\vv{E}_{\text{inc}}, \vv{H}_{\text{inc}})$ are a source-free solution to Maxwell's equations everywhere inside the sphere $S_R$ (since the sources are at infinity). For such fields in a lossless medium,$\avg{\nabla \cdot \vv{S}_{\text{inc}}} = 0$. By the divergence theorem, the flux through any closed surface must be zero (derived in Appendix B, Eq. \eqref{B_Grad_S_inc 3}).
\begin{equation}
    \oint_{S_R}\avg{ \vv{S}_{\text{inc}} \cdot \hat{\vv{r}}} \,dA = \int_{V_R} \avg{(\nabla \cdot \vv{S}_{\text{inc}}) }\,dV = 0
\end{equation}

\paragraph{Scattered Flux:} The scattered fields originate from the particle. The integral of $\vv{S}_{\text{sca}}$ over the sphere gives the total power radiated by the particle in all directions. This is, by definition, the \textbf{scattered power}, $P_{\text{sca}}$.
\begin{equation}\label{2_ Power sca}
    \oint_{S_R} \avg{\vv{S}_{\text{sca}} \cdot \hat{\vv{r}}} \,dA = P_{\text{sca}}
\end{equation}

\paragraph{Interference Flux:} The third integral, involving the cross-terms, represents the interference between the incident and scattered waves. This term accounts for the total power removed from the incident beam by both absorption and scattering. We define the \textbf{extinction power}, $P_{\text{ext}}$, as the negative of this interference flux.
\begin{equation}\label{Power abs}
    \oint_{S_R} \avg{\vv{S}_{\text{int}} \cdot \hat{\vv{r}}} \,dA = -P_{\text{ext}}
\end{equation}
This definition is the basis of the optical theorem, which relates the extinction power to the imaginary part of the forward-scattering amplitude.

\subsection{Assemble the Result}
We now substitute our findings for each of the three flux integrals back into Eq.~\eqref{2_eq:split}:
\begin{equation}
    \oint_{S_R} \avg{\vv{S}} \cdot \hat{\vv{r}} \,dA = 0 + P_{\text{sca}} - P_{\text{ext}}
\end{equation}
Finally, we equate this result with our starting point from the Poynting theorem, Eq.~\eqref{eq:poynting}:
\begin{equation}
    -P_{\text{abs}} = P_{\text{sca}} - P_{\text{ext}}
\end{equation}
Rearranging this equation gives the final, celebrated result:

\begin{equation}\label{2_Power abs}
    \boxed{P_{\text{abs}} = P_{\text{ext}} - P_{\text{sca}}}
\end{equation}
This demonstrates the conservation of energy: the total power removed from the incident beam (\textit{extinction}) is precisely the sum of the power that is re-radiated (\textit{scattering}) and the power that is dissipated as heat (\textit{absorption}).

\section{The Electric-Field Integral Equation (EFIE)}



\subsection{Setup and Conventions}

Time dependence $\sim e^{-i\omega t}$ is assumed. We use Gaussian units and $\mu=1$.
Define $k_0 = \omega/c$ and $\kappa^2(\vec{r}) = \varepsilon(\vec{r}) k_0^2$.
The incident field $\vec{E}_{\text{inc}}$ is what we would have in free space ($\varepsilon=1$) without scatterers; the scattered field is what arises because $\varepsilon(\vec{r}) \neq 1$ in a finite region.
Maxwell's equations are:
\[
\nabla \times \vec{E} = \frac{i\omega}{c} \vec{H}, \qquad \nabla \times \vec{H} = -\frac{i\omega}{c} \, \varepsilon(\vec{r}) \vec{E}. \label{2_ Maxwell eqns}
\]

\subsection{The Vector Wave Equation for $\vec{E}$}

Take the curl of $\nabla \times \vec{E} = (i\omega/c) \vec{H}$:
\[
\nabla \times \nabla \times \vec{E} = \frac{i\omega}{c} \, \nabla \times \vec{H} = -\frac{\omega^2}{c^2} \, \varepsilon(\vec{r}) \vec{E} = -k_0^2 \, \varepsilon(\vec{r}) \vec{E}.
\]
Bringing the RHS to the left:
\begin{equation}
\nabla \times \nabla \times \vec{E} - \varepsilon(\vec{r}) k_0^2 \vec{E} = 0. \label{2_E wave eqn}
\end{equation}


\paragraph{Comment on divergence:} From $\nabla \cdot (\varepsilon \vec{E}) = 0$ (no free charge) we get:
\begin{equation}
\nabla \cdot \vec{E} = -\frac{1}{\varepsilon} \, \vec{E} \cdot \nabla \varepsilon. \label{2_Div E}
\end{equation}
Using the identity $\nabla \times \nabla \times \vec{E} = \nabla(\nabla \cdot \vec{E}) - \nabla^2 \vec{E}$, you can rewrite Eq. \eqref{2_E wave eqn} as:
\begin{equation}
\nabla^2 \vec{E} + \nabla\! \left[ \frac{1}{\varepsilon} \, \vec{E} \cdot \nabla \varepsilon \right] + \varepsilon k_0^2 \vec{E} = 0. \label{2_E wave eqn 2}
\end{equation}
. We won’t actually solve Eq. \eqref{2_E wave eqn 2}; it just tells us the divergence is not zero when $\nabla \varepsilon \neq 0$.

\subsection{Green’s Vector Identity (Levine–Schwinger trick)}

For any smooth $\vec{A}, \vec{B}$ on a volume $V$ with boundary $S$, the identity
\begin{equation}
\int_S d\vec{s} \cdot \big[ \vec{B} \times (\nabla \times \vec{A}) - \vec{A} \times (\nabla \times \vec{B}) \big] = \int_V dV \big[ \vec{A} \cdot (\nabla \times \nabla \times \vec{B}) - \vec{B} \cdot (\nabla \times \nabla \times \vec{A}) \big] \label{2_Levine Schwinger}
\end{equation}
follows from $\nabla \cdot (\vec{C} \times \vec{D}) = \vec{D} \cdot (\nabla \times \vec{C}) - \vec{C} \cdot (\nabla \times \vec{D})$ plus Gauss’ theorem.
We will apply Eq. \eqref{2_Levine Schwinger} with
\begin{align*}
\vec{A}(\vec{r}') &= \boldsymbol{\Gamma}(\vec{r}', \vec{r}) \cdot \vec{e} \quad \text{and} \\
\vec{B}(\vec{r}') &= \vec{E}_{int}(\vec{r}'),
\end{align*}
where $\vec{e}$ is any fixed constant vector and $\boldsymbol{\Gamma}$ is the dyadic (tensor) Green’s function defined next.

\subsection{The Dyadic Green’s Function}

\begin{equation}
\nabla \times \nabla \times \boldsymbol{\Gamma}(\vec{r}, \vec{r}') - k^2 \, \boldsymbol{\Gamma}(\vec{r}, \vec{r}') = \mathbf{I} \, \delta(\vec{r} - \vec{r}'). \label{2_Dyadic Green's func}
\end{equation}
Here $k$ is chosen as the exterior wavenumber (free space), so $k = k_0$. $\mathbf{I}$ is the identity dyad.
Two key identities we will need:

\paragraph{Divergence of Eq. \eqref{2_Dyadic Green's func}.} Taking $\nabla \cdot$ with respect to $\vec{r}$:
\begin{equation}
k^2 \, \nabla \cdot \boldsymbol{\Gamma}(\vec{r}, \vec{r}') = -\nabla \delta(\vec{r} - \vec{r}') = \nabla' \delta(\vec{r} - \vec{r}'). \label{2_Div Dyadic Green's func }
\end{equation}


\paragraph{Link to the scalar Green’s function} $G(\vec{r}, \vec{r}') = \dfrac{e^{ik|\vec{r} - \vec{r}'|}}{4\pi|\vec{r} - \vec{r}'|}$, (from Eq. \eqref{B_Scalar green's function}) which satisfies:
\begin{equation}
(\nabla^2 + k^2)G(\vec{r}, \vec{r}') = -\delta(\vec{r} - \vec{r}'). \label{2_Helmholtz Green func.}
\end{equation}
Using $\nabla \times \nabla \times = -\nabla^2 + \nabla \nabla \cdot$, we find that (From Eq. \eqref{B_eq:dyadic-def} :
\begin{equation}
\boldsymbol{\Gamma}(\vec{r}, \vec{r}') = \left( \mathbf{I} - \frac{1}{k^2} \nabla \nabla' \right) G(\vec{r}, \vec{r}') \quad \text{and} \quad \tilde{\boldsymbol{\Gamma}}(\vec{r}', \vec{r}) = \boldsymbol{\Gamma}(\vec{r}, \vec{r}'). \label{2_Dyadic Green's function def}
\end{equation}
where, \(\nabla \nabla'\) represents the Hessian matrix.

\subsection{Applying the Vector Green's Identity}

We apply the vector Green's identity from the previous section (equation Eq. \eqref{2_Levine Schwinger}) over all space $V = \mathbb{R}^3$. We choose our vectors to be $\vec{A}(\vec{r}') = \boldsymbol{\Gamma}(\vec{r}', \vec{r}) \cdot \vec{e}$ and $\vec{B}(\vec{r}') = \vec{E}_{int}(\vec{r}')$. The core idea is that the surface integral at infinity will yield the incident field, while the volume integral will relate the total field $\vec{E}$ to the scattering source.

\paragraph{Right-Hand Side (Volume Integral):} We substitute the wave equations for $\vec{E}$ and $\boldsymbol{\Gamma}$:
\begin{itemize}
    \item From Eq. \eqref{2_E wave eqn}: $\nabla' \times \nabla' \times \vec{E}_{int}(\vec{r}') = \varepsilon(\vec{r}') k_0^2 \vec{E}_{int}(\vec{r}')$
    \item From Eq. \eqref{2_Dyadic Green's func} : $\nabla' \times \nabla' \times \boldsymbol{\Gamma}(\vec{r}', \vec{r}) = k^2 \boldsymbol{\Gamma}(\vec{r}', \vec{r}) + \mathbf{I} \delta(\vec{r}' - \vec{r})$
\end{itemize}
The volume integral becomes:
\begin{align*}
    &\int_{\mathbb{R}^3} dV' \Big[ (\boldsymbol{\Gamma} \cdot \vec{e}) \cdot (\nabla' \times \nabla' \times \vec{E}) - \vec{E} \cdot (\nabla' \times \nabla' \times (\boldsymbol{\Gamma} \cdot \vec{e})) \Big] \\
    &= \int_{\mathbb{R}^3} dV' \Big[ (\boldsymbol{\Gamma} \cdot \vec{e}) \cdot (\varepsilon k_0^2 \vec{E}) - \vec{E} \cdot \bigg( k^2(\boldsymbol{\Gamma} \cdot \vec{e}) + \mathbf{I} \delta(\vec{r}'-\vec{r}) \cdot \vec{e} \bigg) \Big] \\
    &= \int_{V} dV' \, [\varepsilon(\vec{r}') k_0^2 - k^2] \, \vec{E}_{int}(\vec{r}') \cdot \boldsymbol{\Gamma}(\vec{r}', \vec{r}) \cdot \vec{e} - \int_{\mathbb{R}^3} dV' \, \vec{E}_{int}(\vec{r}') \cdot \mathbf{I} \delta(\vec{r}'-\vec{r}) \cdot \vec{e} \\
    &= \int_{V} dV' \, [\varepsilon(\vec{r}') k_0^2 - k^2] \, \vec{E}_{int}(\vec{r}') \cdot \boldsymbol{\Gamma}(\vec{r}', \vec{r}) \cdot \vec{e} - \vec{E}(\vec{r}) \cdot \vec{e}
\end{align*}
Here, the integral domain $V$ can be restricted to the region where $\varepsilon(\vec{r}') \neq 1$.

\paragraph{Left-Hand Side (Surface Integral):} With an outgoing Green's function, the surface integral at infinity gives the incident field: $\int_S \dots = -\vec{E}_{\text{inc}}(\vec{r}) \cdot \vec{e}$.

\paragraph{Combining Both Sides:} Equating the left and right sides and rearranging gives:
\begin{equation}
\vec{E}(\vec{r}) \cdot \vec{e} = \vec{E}_{\text{inc}}(\vec{r}) \cdot \vec{e} + \int_V dV' \, \Big[\varepsilon(\vec{r}')k_0^2 - k^2\Big] \, \vec{E}_{int}(\vec{r}') \cdot \boldsymbol{\Gamma}(\vec{r}', \vec{r}) \cdot \vec{e}. \label{2_ Levine E eqn}
\end{equation}
We choose the Green's function for free space, so $k=k_0$. The term in brackets simplifies to $[\varepsilon(\vec{r}')-1]k_0^2$. Since the constant vector $\vec{e}$ is arbitrary, we can write the dyadic equation:
\begin{equation}
\boxed{\;
\vec{E}(\vec{r}) = \vec{E}_{\text{inc}}(\vec{r}) + \int_V k_0^2 \, [\varepsilon(\vec{r}') - 1] \, \boldsymbol{\Gamma}(\vec{r}, \vec{r}') \cdot \vec{E}_{int}(\vec{r}') \, dV'.
\;} \label{2_ Levine E eqn 2}
\end{equation}
This famous result is the \textbf{Lippmann-Schwinger equation} for vector fields. It states that the total field is the incident field plus the field radiated by the equivalent polarization currents $\vec{J}_{\text{eq}}(\vec{r}') \equiv k_0^2[\varepsilon(\vec{r}')-1]\vec{E}_{int}(\vec{r}')$.

\subsection{Derivation of the Surface Integral Term}
Let's walk through the derivation of Eq. \eqref{2_ Levine E eqn}  more slowly to see exactly where the incident field term comes from. 

\subsubsection{Goal}
Starting from Green’s vector identity, show that:
\[
\boxed{\;
\vec{E}(\vec{r})\!\cdot\!\vec{e} = \vec{E}_{\text{inc}}(\vec{r})\!\cdot\!\vec{e} + \int_V\!\big[\varepsilon(\vec{r}')k_0^2-k^2\big]\, \vec{E}_{int}(\vec{r}')\!\cdot\!\boldsymbol{\Gamma}(\vec{r}', \vec{r})\!\cdot\!\vec{e}\;dV'.
\;}
\]
Here $k_0 = \omega/c$, $k$ is the wavenumber used to build $\boldsymbol{\Gamma}$, and $\boldsymbol{\Gamma}$ solves:
\begin{equation}
\nabla'\!\times\nabla'\!\times \boldsymbol{\Gamma}(\vec{r}', \vec{r}) - k^2\boldsymbol{\Gamma}(\vec{r}', \vec{r}) = \mathbf{I}\,\delta(\vec{r}'-\vec{r}). \label{2_Green's func diff eqn}
\end{equation}

\subsubsection{Start from Green’s Identity}
The identity (with derivatives w.r.t. $\vec{r}'$) is:
\begin{equation}
\int_{S} \!\! d\vec{s}'\!\cdot\!\Big[\vec{B}\times(\nabla'\!\times\vec{A})-\vec{A}\times(\nabla'\!\times\vec{B})\Big] = \int_{V}\!\! dV'\,\Big[\vec{A}\!\cdot\!(\nabla'\!\times\nabla'\!\times\vec{B})-\vec{B}\!\cdot\!(\nabla'\!\times\nabla'\!\times\vec{A})\Big]. \label{2_ Levine identity}
\end{equation}
Choose $\vec{A}(\vec{r}') = \boldsymbol{\Gamma}(\vec{r}', \vec{r}) \cdot \vec{e}$ and $\vec{B}(\vec{r}') = \vec{E}_{int}(\vec{r}')$.

\subsubsection{Evaluate the Volume (RHS) Integral}
We use the wave equations for the fields inside the integral:
\begin{itemize}
    \item Maxwell's equations give: $\nabla'\!\times\nabla'\!\times\vec{E}_{int}(\vec{r}')=\varepsilon(\vec{r}')k_0^2\,\vec{E}_{int}(\vec{r}')$.
    \item The definition of $\boldsymbol{\Gamma}$ gives: $\nabla'\!\times\nabla'\!\times(\boldsymbol{\Gamma}\cdot\vec{e})=k^2(\boldsymbol{\Gamma}\cdot\vec{e})+\vec{e}\,\delta(\vec{r}'-\vec{r})$.
\end{itemize}
Plugging these into the RHS of Eq. \eqref{2_ Levine identity}  gives:
\begin{align}
    &\int_V\!\! dV'\,\Big[(\boldsymbol{\Gamma}\cdot\vec{e})\!\cdot\!(\varepsilon k_0^2\vec{E}) - \vec{E}\!\cdot\!\big(k^2\boldsymbol{\Gamma}\cdot\vec{e}+\vec{e}\,\delta\big)\Big] \nonumber \\
    &= \int_V\!\! dV'\,\big(\varepsilon(\vec{r}') k_0^2-k^2\big)\,\vec{E}_{int}(\vec{r}')\!\cdot\!(\boldsymbol{\Gamma}(\vec{r}',\vec{r})\cdot\vec{e}) - \vec{E}(\vec{r})\!\cdot\!\vec{e}. \label{2_Levine E RHS 1}
\end{align}
So, Green's identity now reads:
\begin{equation}
\underbrace{\int_{S}\!\! d\vec{s}'\!\cdot\!\Big[\vec{E}\times(\nabla'\!\times(\boldsymbol{\Gamma}\cdot\vec{e})) - (\boldsymbol{\Gamma}\cdot\vec{e})\times(\nabla'\!\times\vec{E})\Big]}_{\displaystyle \mathcal{S}} = \int_V\!\!\big(\varepsilon k_0^2-k^2\big)\,\vec{E}\!\cdot\!(\boldsymbol{\Gamma}\cdot\vec{e})\,dV' - \vec{E}(\vec{r})\!\cdot\!\vec{e}. \label{2_Levine E RHS 2}
\end{equation}
Rearranging for $\vec{E}(\vec{r})\cdot\vec{e}$ gives:
\begin{equation}
\vec{E}(\vec{r})\!\cdot\!\vec{e} = \int_V\!\big(\varepsilon k_0^2-k^2\big)\,\vec{E}\!\cdot\!(\boldsymbol{\Gamma}\cdot\vec{e})\,dV' - \mathcal{S}. \label{2_Levine E RHS 3}
\end{equation}
To finish, we must show that the surface term $\mathcal{S}$ is equal to $-\vec{E}_{\text{inc}}(\vec{r}) \cdot \vec{e}$.

\subsubsection{Evaluate the Surface Term $\mathcal{S}$}
Let the total field be $\vec{E} = \vec{E}_{\text{inc}} + \vec{E}_{\text{sct}}$. The surface integral splits into two parts: $\mathcal{S} = \mathcal{S}_{\text{inc}} + \mathcal{S}_{\text{sct}}$.
\begin{itemize}
    \item[(a)] \textbf{The scattered part vanishes.} Both $\vec{E}_{\text{sct}}$ and $\boldsymbol{\Gamma}$ represent outgoing waves. They obey the Sommerfeld radiation condition, meaning they decay as $e^{ikR}/R$ at large distances $R$. This causes the integrand to decay faster than the surface area ($R^2$) grows, so $\mathcal{S}_{\text{sct}} \to 0$ as the surface $S$ is taken to infinity.
    \item[(b)] \textbf{The incident part gives the result.} Let's evaluate $\mathcal{S}_{\text{inc}}$ by applying Green's identity a second time, but for the incident field $\vec{E}_{\text{inc}}$ and $\boldsymbol{\Gamma}$ in a volume free of scatterers (where $\varepsilon=1$ and $k=k_0$). The volume integral is:
    \[
    \int_V\!\! dV'\,\Big[\vec{E}_{\text{inc}}\!\cdot\!(\nabla'\!\times\nabla'\!\times(\boldsymbol{\Gamma}\cdot\vec{e})) - (\boldsymbol{\Gamma}\cdot\vec{e})\!\cdot\!(\nabla'\!\times\nabla'\!\times\vec{E}_{\text{inc}})\Big].
    \]
    In free space, $\nabla'\!\times\nabla'\!\times\vec{E}_{\text{inc}} = k^2\vec{E}_{\text{inc}}$, and from Eq. \eqref{2_Green's func diff eqn} , $\nabla'\!\times\nabla'\!\times(\boldsymbol{\Gamma}\cdot\vec{e}) = k^2(\boldsymbol{\Gamma}\cdot\vec{e}) + \vec{e}\delta$. The $k^2$ terms cancel perfectly! The volume integral simplifies to:
    \[
    \int_V\!\! dV'\,\vec{E}_{\text{inc}}(\vec{r}') \!\cdot\! \left( \vec{e} \delta(\vec{r}'-\vec{r}) \right) = \vec{E}_{\text{inc}}(\vec{r})\!\cdot\!\vec{e}.
    \]
    This volume integral is equal to the surface integral from Green's identity:
    \[
    \int_{S}\!\! d\vec{s}'\!\cdot\!\Big[\vec{E}_{\text{inc}}\times(\nabla'\!\times(\boldsymbol{\Gamma}\cdot\vec{e})) - (\boldsymbol{\Gamma}\cdot\vec{e})\times(\nabla'\!\times\vec{E}_{\text{inc}})\Big] = \vec{E}_{\text{inc}}(\vec{r})\!\cdot\!\vec{e}.
    \]
    Notice the terms in the bracket are in the opposite order of our original definition of $\mathcal{S}_{\text{inc}}$. Therefore, $\mathcal{S}_{\text{inc}} = -\vec{E}_{\text{inc}}(\vec{r})\!\cdot\!\vec{e}$.
\end{itemize}
Combining the terms above, we find the total surface term is $\mathcal{S} = \mathcal{S}_{\text{inc}} + \mathcal{S}_{\text{sct}} = -\vec{E}_{\text{inc}}(\vec{r})\!\cdot\!\vec{e}$.

\subsubsection{Final Assembly}
Inserting our result for $\mathcal{S}$ back into equation Eq. \eqref{2_Levine E RHS 3} , we get:
\[
\vec{E}(\vec{r})\!\cdot\!\vec{e} = \int_V\!\big(\varepsilon k_0^2-k^2\big)\,\vec{E}_{int}(\vec{r}')\!\cdot\!\boldsymbol{\Gamma}(\vec{r}',\vec{r})\!\cdot\!\vec{e}\,dV' - (-\vec{E}_{\text{inc}}(\vec{r})\!\cdot\!\vec{e}),
\]
which is exactly the target equation Eq. \eqref{2_ Levine E eqn} .

\subsection{Electric field in terms of Scalar Green’s Function}

We now substitute the expression for $\boldsymbol{\Gamma}$ from Eq. \eqref{2_Dyadic Green's function def}  into our final result Eq. \eqref{2_ Levine E eqn 2} :
\[
\boldsymbol{\Gamma}(\vec{r}, \vec{r}') = \left(\mathbf{I} - \frac{1}{k_0^2}\nabla\nabla'\right) G(\vec{r}, \vec{r}').
\]
The integral becomes, with \(\vec{E}_{int}(\vec{r}')\) representing the Electric field inside the volume of the sphere:
\begin{align}
\vec{E}(\vec{r}) &= \vec{E}_{\text{inc}}(\vec{r}) + \int_V k_0^2 \, [\varepsilon(\vec{r}')-1] \, \left(\mathbf{I} - \frac{1}{k_0^2}\nabla\nabla'\right) G(\vec{r}, \vec{r}') \cdot \vec{E}_{int}(\vec{r}') \, dV' \nonumber \\
&= \vec{E}_{\text{inc}}(\vec{r}) + k_0^2\int_V [\varepsilon(\vec{r}')-1]\, G(\vec{r}, \vec{r}')\,\vec{E}_{int}(\vec{r}')\,dV' \nonumber \\
&\quad - \int_V [\varepsilon(\vec{r}')-1] \, (\nabla\nabla' G(\vec{r}, \vec{r}')) \cdot \vec{E}_{int}(\vec{r}') \, dV'.
\end{align}
Since $\nabla$ acts on $\vec{r}$ and not $\vec{r}'$, we can pull it outside the integral:
\begin{equation}
\begin{aligned}
\vec{E}(\vec{r}) &= \vec{E}_{\text{inc}}(\vec{r}) + k_0^2\int_V [\varepsilon(\vec{r}')-1]\, G(\vec{r}, \vec{r}')\,\vec{E}_{int}(\vec{r}')\,dV' \\
&\quad - \nabla \int_V [\varepsilon(\vec{r}')-1]\, \big[\nabla'G(\vec{r}, \vec{r}') \cdot \vec{E}_{int}(\vec{r}')\big]\,dV'.
\end{aligned} \label{2_E greens func}
\end{equation}
This is the electric-field volume integral equation in its scalar-Green form. The outgoing boundary condition is automatically satisfied because it's built into $G$.



\subsection{Far-Field Asymptotics of $G$ and $\nabla G$}

Let $\hat{s} := \vec{r}/r$ be the unit vector pointing from the origin (assumed to be inside the scatterer) to the observer. The scatterer is contained in a bounded volume $V$, so for any source point $\vec{r}' \in V$, we have $r \gg r' = |\vec{r}'|$. We can then approximate the distance $|\vec{r}-\vec{r}'|$:
\[
|\vec{r}-\vec{r}'| = \sqrt{(\vec{r}-\vec{r}') \cdot (\vec{r}-\vec{r}')} = \sqrt{r^2 - 2\vec{r}\cdot\vec{r}' + r'^2} = r \sqrt{1 - \frac{2\hat{s}\cdot\vec{r}'}{r} + \frac{r'^2}{r^2}} \approx r - \hat{s}\cdot\vec{r}'.
\]
In the denominator of $G$, we can simply use $|\vec{r}-\vec{r}'| \approx r$. In the exponent (the phase), the second term is crucial. This gives the far-field approximation for $G$:
\begin{equation}
G(\vec{r},\vec{r}') = \frac{e^{ik|\vec{r}-\vec{r}'|}}{4\pi|\vec{r}-\vec{r}'|} \;\approx\; \frac{e^{ik(r - \hat{s}\cdot\vec{r}')}}{4\pi r} = \frac{e^{ikr}}{4\pi r}\,e^{-ik\,\hat{s}\cdot \vec{r}'}. \label{2_Scalar green's func approx}
\end{equation}
Next, we need the gradients. For $\nabla$ (derivatives w.r.t. $\vec{r}$), the dominant contribution comes from differentiating the rapidly oscillating $e^{ikr}$ term:
\[
\nabla \left(\frac{e^{ikr}}{4\pi r}\right) = \hat{s} \frac{d}{dr}\left(\frac{e^{ikr}}{4\pi r}\right) = \hat{s} \left(\frac{ik e^{ikr}}{4\pi r} - \frac{e^{ikr}}{4\pi r^2}\right) = ik\,\hat{s}\,\frac{e^{ikr}}{4\pi r} \left(1 + \frac{i}{kr}\right).
\]
In the far field ($kr \gg 1$), we keep only the leading order term in $1/r$:
\begin{equation}
\nabla G(\vec{r},\vec{r}') \approx \nabla \left(\frac{e^{ikr}}{4\pi r}\right) e^{-ik\,\hat{s}\cdot \vec{r}'} \approx ik\,\hat{s}\,\frac{e^{ikr}}{4\pi r} e^{-ik\,\hat{s}\cdot \vec{r}'}. \label{2_Grad Scalar green's func approx}
\end{equation}
Similarly, for $\nabla'$ (derivatives w.r.t. $\vec{r}'$), the only term that depends on $\vec{r}'$ in the far-field form is the phase $e^{-ik\,\hat{s}\cdot \vec{r}'}$.
\[
\nabla' e^{-ik\,\hat{s}\cdot \vec{r}'} = -ik\,\hat{s}e^{-ik\,\hat{s}\cdot \vec{r}'}.
\]
So, the gradient with respect to the source coordinates is:
\begin{equation}
\nabla' G(\vec{r},\vec{r}') \approx \frac{e^{ikr}}{4\pi r} \nabla' \left( e^{-ik\,\hat{s}\cdot \vec{r}'} \right) \approx -ik\,\hat{s}\, \frac{e^{ikr}}{4\pi r}\,e^{-ik\,\hat{s}\cdot \vec{r}'}. \label{2_Grad Scalar green's func approx 2}
\end{equation}

\subsection{Evaluating the First Volume Integral in the Far Field}
Using approximation Eq. \eqref{2_Scalar green's func approx}  for $G$ in the first integral of Eq. \eqref{2_E greens func} :
\begin{align}
k^2\!\int_V [\varepsilon(\vec{r}')-1]\,\vec{E}_{int}(\vec{r}')\,G\,dV' &\approx k^2\int_V [\varepsilon(\vec{r}')-1]\,\vec{E}_{int}(\vec{r}') \left(\frac{e^{ikr}}{4\pi r}\,e^{-ik\,\hat{s}\cdot \vec{r}'}\right) dV' \nonumber \\
&\approx \frac{e^{ikr}}{r}\,\frac{k^2}{4\pi} \int_V e^{-ik\,\hat{s}\cdot \vec{r}'}[\varepsilon(\vec{r}')-1]\,\vec{E}_{int}(\vec{r}')\,dV' \nonumber \\
&=: \frac{e^{ikr}}{r}\,\vec{P}(\hat{s}). \label{2_Sub approx in EFIE}
\end{align}
This defines the vector amplitude $\vec{P}(\hat{s})$, which is essentially the Fourier transform of the equivalent polarisation current. If we assume the internal electric field isn't dependent on the position inside the sphere, it has a fixed direction, then, \(\vec{P}(\hat{s})\) in the direction of the internal electric field \(\vec{E}_{int}(\vec{r}')\):
\begin{equation}
\boxed{\;\displaystyle
\vec{P}(\hat{s}) = \frac{k^2}{4\pi}\int_V e^{-ik\,\hat{s}\cdot \vec{r}'}\,[\varepsilon(\vec{r}')-1]\;\vec{E}_{int}(\vec{r}')\,dV'.
} \label{2_ E field vector amplitude}
\end{equation}

\subsection{Evaluating the Second (Gradient) Term in the Far Field}
Let's call the scalar integral inside the gradient term $I(\vec{r})$:
\begin{equation}
I(\vec{r}) := \int_V \big[\nabla' G(\vec{r},\vec{r}')\cdot \vec{E}_{int}(\vec{r}')\big]\,[\varepsilon(\vec{r}')-1]\,dV'. 
\end{equation}
Using approximation Eq. \eqref{2_Grad Scalar green's func approx 2}  for $\nabla'G$:
\begin{align}
I(\vec{r}) &\approx \int_V \left[\left(-ik\,\hat{s}\, \frac{e^{ikr}}{4\pi r}\,e^{-ik\,\hat{s}\cdot \vec{r}'}\right)\cdot \vec{E}_{int}(\vec{r}')\right]\,[\varepsilon(\vec{r}')-1]\,dV' \nonumber \\
&= -\frac{ik\,e^{ikr}}{4\pi r}\,\hat{s}\cdot \int_V e^{-ik\,\hat{s}\cdot \vec{r}'}\,[\varepsilon(\vec{r}')-1]\;\vec{E}_{int}(\vec{r}')\,dV' \nonumber \\
&= -\frac{ik\,e^{ikr}}{r} \left(\frac{1}{k^2}\right) \hat{s} \cdot \vec{P}(\hat{s}) = -\frac{e^{ikr}}{r}\,\frac{i}{k}\;\hat{s}\cdot \vec{P}(\hat{s}). 
\end{align}
Now we apply $-\nabla$ to this result. Using approximation Eq. \eqref{2_Grad Scalar green's func approx}  for the gradient of the outgoing spherical wave:
\begin{align}
-\nabla I(\vec{r}) &\approx -\nabla\!\left[\frac{e^{ikr}}{r}\right]\;\left(-\frac{i}{k}\,\hat{s}\cdot \vec{P}(\hat{s})\right) \nonumber \\
&\approx -\left(ik\,\hat{s}\,\frac{e^{ikr}}{r}\right)\;\left(-\frac{i}{k}\,\hat{s}\cdot \vec{P}(\hat{s})\right) \nonumber \\
&= -\frac{e^{ikr}}{r}\;\hat{s}\,\big(\hat{s}\cdot \vec{P}(\hat{s})\big).  \label{2_Sub approx in EFIE 2}
\end{align}

\subsection{Combining the Pieces to Find the Scattering Electric Field}
The total scattered field $\vec{E}_{\text{sctd}}$ is the sum of the results from (Eq. \eqref{2_Sub approx in EFIE}) and (Eq. \eqref{2_Sub approx in EFIE 2}):
\begin{align*}
\vec{E}_{\text{sctd}}(\vec{r}) &\approx \frac{e^{ikr}}{r}\,\vec{P}(\hat{s}) - \frac{e^{ikr}}{r}\;\hat{s}\,\big(\hat{s}\cdot \vec{P}(\hat{s})\big) \\
&= \frac{e^{ikr}}{r}\,\Big[\,\vec{P}(\hat{s})\;-\;\hat{s}\,\big(\hat{s}\cdot \vec{P}(\hat{s})\big)\,\Big].
\end{align*}
This gives the final form for the scattered field:
\begin{equation}
\boxed{\;\displaystyle
\vec{E}_{\text{sctd}}(\vec{r})\;\approx\; \frac{e^{ikr}}{r}\,\Big[\,\vec{P}(\hat{s})\;-\;\hat{s}\,\big(\hat{s}\cdot \vec{P}(\hat{s})\big)\,\Big].
\;} 
\end{equation}
The term in the brackets is the component of the vector $\vec{D}$ that is transverse to the direction of propagation $\hat{s}$, which is exactly what we expect for an electromagnetic wave. We can express this using the transverse projector dyadic $\mathbf{P}_\perp(\hat{s}) = \mathbf{I} - \hat{s}\hat{s}^{\top}$:
\begin{equation}
\vec{E}_{\text{sctd}}(\vec{r}) \approx \frac{e^{ikr}}{r}\;\mathbf{P}_\perp(\hat{s})\,\vec{P}(\hat{s}). 
\end{equation}
A more common way to write this transverse projection is using the vector triple product identity $\vec{A} - \hat{s}(\hat{s}\cdot\vec{A}) = \hat{s} \times (\vec{A} \times \hat{s})$. Applying this, we arrive at the standard form for the scattered field in terms of the scattering amplitude $\vec{F}_1$:
\begin{equation}
\boxed{\;\displaystyle
\vec{E}_{\text{sctd}}(\vec{r})\;\approx\;\frac{e^{ikr}}{r}\;\vec{F}_1(\hat{s})\quad \text{with}\quad
\vec{F}_1(\hat{s})=\hat{s}\times\big(\vec{P}(\hat{s}) \times \hat{s}\big),
\;} 
\end{equation}
where \(
\vec{P}(\hat{s}) = \frac{k^2}{4\pi}\int_V e^{-ik\,\hat{s}\cdot \vec{r}'}\,[\varepsilon(\vec{r}')-1]\;\vec{E}_{int}(\vec{r}')\,dV'
 \) and $\hat{s}:= \vec{r}/r$ is the unit vector pointing from the origin (assumed to be inside the scatterer) to the observer. 
\paragraph{Remarks:}
\begin{itemize}
    \item The direction of \(\vec{F}_1(\hat{s})\) is dependent on the direction of \(\vec{P}(\hat{s})\) which is in the direction of the internal electric field \(\vec{E}_{int}(\vec{r}')\). This is if we assume that the internal electric field isn't dependent on the position of the point inside the sphere and is constant.
    \item The final result is manifestly transverse, as $\hat{s} \cdot \vec{F}_1 = \hat{s} \cdot (\hat{s} \times (\dots)) = 0$.

\end{itemize}


\section{The Magnetic-Field Integral Equation (MFIE)}
\subsection{Setup and Conventions}
We use Gaussian units with $\mu=1$. The time-dependence is $e^{-i\omega t}$. The free-space wavenumber is $k_0 = \omega/c$, and the position-dependent wavenumber is $\kappa^2(\vec{r}) = \varepsilon(\vec{r}) k_0^2$. The dyadic Green's function for free space, $\boldsymbol{\Gamma}$, satisfies:
\[
\nabla\times\nabla\times\boldsymbol{\Gamma}(\vec{r},\vec{r}') - k_0^2\,\boldsymbol{\Gamma}(\vec{r},\vec{r}') = \mathbf{I}\,\delta(\vec{r}-\vec{r}'),
\]
and can be expressed in terms of the scalar Green's function $G(\vec{r},\vec{r}')=\dfrac{e^{ik_0|\vec{r}-\vec{r}'|}}{4\pi|\vec{r}-\vec{r}'|}$ as:
\[
\boldsymbol{\Gamma}(\vec{r},\vec{r}') = \left(\mathbf{I} - \frac{1}{k_0^2}\nabla\nabla'\right)G(\vec{r},\vec{r}').
\]


\subsection{Rewrite the Magnetic Wave Equation into "Source" Form}


We start from the second-order wave equation for the magnetic field $\vec{H}$ in an inhomogeneous medium (Eq. \eqref{2_ Maxwell eqns}):
\[
\nabla\times\!\left[\frac{1}{\varepsilon(\vec{r})}\,\nabla\times\vec{H}(\vec{r})\right]-k_0^2\,\vec{H}(\vec{r})=0.
\]
Our first goal is to rearrange this into a form that looks like a standard Helmholtz equation with an explicit source term. We begin by applying the vector product rule $\nabla \times (f\vec{A}) = f(\nabla \times \vec{A}) + (\nabla f) \times \vec{A}$:
\[
\nabla\times\!\left(\frac{1}{\varepsilon}\,\nabla\times\vec{H}\right) = \frac{1}{\varepsilon}\,\nabla\times\nabla\times\vec{H} + \nabla\!\left(\frac{1}{\varepsilon}\right)\times(\nabla\times\vec{H}).
\]
Next, we simplify the two terms on the right. For the second term, we use Maxwell's equation for the curl of $\vec{H}$ and the chain rule for $\nabla(1/\varepsilon)$:
\begin{itemize}
    \item $\nabla\times\vec{H} = -\dfrac{i\omega}{c}\,\varepsilon\,\vec{E} = -ik_0\,\varepsilon\,\vec{E}$
    \item $\nabla(1/\varepsilon) = -\dfrac{\nabla\varepsilon}{\varepsilon^2}$
\end{itemize}
Substituting these in gives:
\[
\nabla\!\left(\frac{1}{\varepsilon}\right)\times(\nabla\times\vec{H}) = -\frac{\nabla\varepsilon}{\varepsilon^2}\times\big(-ik_0\,\varepsilon\,\vec{E}\big) = \frac{ik_0}{\varepsilon}\,\nabla\varepsilon\times\vec{E}.
\]
Plugging this result back into the wave equation and multiplying the entire equation by $\varepsilon(\vec{r})$ yields:
\[
\nabla\times\nabla\times\vec{H} + ik_0\,\nabla\varepsilon(\vec{r})\times\vec{E}(\vec{r}) - \varepsilon(\vec{r})k_0^2\,\vec{H}(\vec{r}) = 0.
\]
Rearranging this to isolate the Helmholtz operator acting on $\vec{H}$ (and recalling $\kappa^2 = \varepsilon k_0^2$), we get our desired "source" form:
\begin{equation}
\boxed{\;
\nabla\times\nabla\times\vec{H}(\vec{r}) - \kappa^2(\vec{r})\vec{H}(\vec{r}) = -ik_0\,\nabla\varepsilon(\vec{r})\times\vec{E}(\vec{r}).
\;}
\end{equation}
This powerful result shows that for the magnetic field, the "source" driving the scattering is not just the presence of the dielectric ($\varepsilon \neq 1$), but specifically the \textit{gradient} of the permittivity, $\nabla\varepsilon$.

\subsection{Obtain the Volume Integral Equation via Green's Identity}


We now use the Levine-Schwinger vector Green's identity to convert our differential equation into an integral equation. The identity is:
\[
\int_S d\vec{s}'\cdot\big[\vec{B}\times(\nabla'\times\vec{A})-\vec{A}\times(\nabla'\times\vec{B})\big] = \int_V dV'\,\big[\vec{A}\cdot(\nabla'\times\nabla'\times\vec{B})-\vec{B}\cdot(\nabla'\times\nabla'\times\vec{A})\big].
\]
We make the following choices for our vector fields:
\[
\vec{A}(\vec{r}') = \boldsymbol{\Gamma}(\vec{r}',\vec{r})\cdot\vec{e}, \qquad \vec{B}(\vec{r}') = \vec{H}(\vec{r}'),
\]
where $\vec{e}$ is an arbitrary constant vector.

\paragraph{Right-Hand Side (Volume Term).} We substitute our "source" equation for $\nabla'\times\nabla'\times\vec{H}$ and the defining equation for $\boldsymbol{\Gamma}$:
\begin{align*}
\nabla'\times\nabla'\times\vec{H}(\vec{r}') &= \kappa^2(\vec{r}')\vec{H}(\vec{r}') - ik_0\,\nabla'\varepsilon(\vec{r}')\times\vec{E}_{int}(\vec{r}'), \\
\nabla'\times\nabla'\times(\boldsymbol{\Gamma}\cdot\vec{e}) &= k_0^2(\boldsymbol{\Gamma}\cdot\vec{e}) + \vec{e}\,\delta(\vec{r}'-\vec{r}).
\end{align*}
The volume integral becomes:
\begin{align*}
\text{RHS} &= \int_V\!\Big[(\boldsymbol{\Gamma}\cdot\vec{e})\cdot(\kappa^2\vec{H}-ik_0\,\nabla'\varepsilon\times\vec{E}) - \vec{H}\cdot\big(k_0^2(\boldsymbol{\Gamma}\cdot\vec{e})+\vec{e}\,\delta\big)\Big]\,dV' \\
&= \int_V\!(\kappa^2-k_0^2)\,\vec{H}\cdot(\boldsymbol{\Gamma}\cdot\vec{e})\,dV' - ik_0\!\int_V\!(\boldsymbol{\Gamma}\cdot\vec{e})\cdot(\nabla'\varepsilon\times\vec{E})\,dV' - \vec{H}(\vec{r})\cdot\vec{e}.
\end{align*}

\paragraph{Left-Hand Side (Surface Term).} Just as in the electric-field case, the surface integral at infinity, $\mathcal{S}_H$, evaluates to the incident field. The scattered field $\vec{H}_{\text{sct}}$ satisfies the Sommerfeld radiation condition, causing its contribution to vanish. A second application of Green's identity shows that the incident field part gives $\mathcal{S}_H^{(\text{inc})} = -\vec{H}_{\text{inc}}(\vec{r})\cdot\vec{e}$.

\paragraph{Combining and Solving.} Equating the LHS and RHS and rearranging to solve for $\vec{H}(\vec{r})\cdot\vec{e}$, we find:
\[
\vec{H}(\vec{r})\cdot\vec{e} = \vec{H}_{\text{inc}}(\vec{r})\cdot\vec{e} + \int_V(\kappa^2-k_0^2)\,\vec{H}\cdot(\boldsymbol{\Gamma}\cdot\vec{e})\,dV' - ik_0\int_V(\boldsymbol{\Gamma}\cdot\vec{e})\cdot(\nabla'\varepsilon\times\vec{E})\,dV'.
\]
Since $\vec{e}$ is arbitrary, we can promote this to a vector equation. Using $\kappa^2-k_0^2 = k_0^2(\varepsilon-1)$, we arrive at the final Magnetic-Field Integral Equation:
\begin{equation}
\boxed{\;
\begin{aligned}
\vec{H}(\vec{r}) = \vec{H}_{\text{inc}}(\vec{r}) &+ \int_V k_0^2\,(\varepsilon(\vec{r}')-1)\,\boldsymbol{\Gamma}(\vec{r},\vec{r}')\cdot\vec{H}(\vec{r}')\,dV' \\
&-ik_0\int_V \boldsymbol{\Gamma}(\vec{r},\vec{r}')\cdot\big[\nabla'\varepsilon(\vec{r}')\times\vec{E}_{int}(\vec{r}')\big]\,dV'.
\end{aligned}
\;}
\end{equation}


\subsection{The Far-Field Limit and Scattering Amplitude $\vec{A}_2$}


To find the scattered field at a large distance ($r=|\vec{r}|\to\infty$), we use the asymptotic form of the Green's function. Let $\hat{s}=\vec{r}/r$.
\[
G(\vec{r},\vec{r}') \approx \frac{e^{ik_0 r}}{4\pi r}\,e^{-ik_0\,\hat{s}\cdot\vec{r}'} \qquad \text{and} \qquad \nabla\nabla'G \approx (-ik_0)^2 \hat{s}\hat{s}^{\top} G.
\]
This leads to the far-field approximation for the dyadic Green's function:
\[
\boldsymbol{\Gamma}(\vec{r},\vec{r}') = \Big(\mathbf{I}-\frac{1}{k_0^2}\nabla\nabla'\Big)G \xrightarrow{\text{far field}} \Big(\mathbf{I}-\hat{s}\hat{s}^{\top}\Big)\,\frac{e^{ik_0 r}}{4\pi r}\,e^{-ik_0\,\hat{s}\cdot\vec{r}'} = \mathbf{P}_\perp(\hat{s}) G(\vec{r},\vec{r}').
\]
Applying this to the two integrals in the MFIE:
\begin{itemize}
    \item \textbf{First Term:}
    \[
    \int_V k_0^2(\varepsilon-1)\,\boldsymbol{\Gamma}\cdot\vec{H}\,dV' \approx \frac{e^{ik_0 r}}{r}\;\mathbf{P}_\perp(\hat{s})\, \underbrace{\left[\frac{k_0^2}{4\pi}\int_V e^{-ik_0\,\hat{s}\cdot\vec{r}'}\,(\varepsilon-1)\,\vec{H}(\vec{r}')\,dV'\right]}_{:=\ \vec{P}'(\hat{s})}.
    \]
    \item \textbf{Second Term:}
    \[
    -ik_0\int_V \boldsymbol{\Gamma}\cdot\big(\nabla'\varepsilon\times\vec{E}\big)\,dV' \approx \frac{e^{ik_0 r}}{r}\;\mathbf{P}_\perp(\hat{s})\, \underbrace{\left[-\frac{ik_0}{4\pi}\int_V e^{-ik_0\,\hat{s}\cdot\vec{r}'}\,\nabla'\varepsilon(\vec{r}')\times\vec{E}_{int}(\vec{r}')\,dV'\right]}_{:=\ \vec{P}''(\hat{s})}.
    \]
\end{itemize}
The total scattered field $\vec{H}_{\text{sctd}} = \vec{H} - \vec{H}_{\text{inc}}$ is the sum of these two contributions:
\begin{equation}
\boxed{\;
\vec{H}_{\text{sctd}}(\vec{r})\ \approx\ \frac{e^{ik_0 r}}{r}\;\mathbf{P}_\perp(\hat{s})\,[\,\vec{P}'(\hat{s})+\vec{P}''(\hat{s})\,].
\;}
\end{equation}
Using the identity $\mathbf{P}_\perp(\hat{s})\vec{A} = -\hat{s}\times(\hat{s}\times\vec{A})$, we can define the magnetic scattering amplitude $\vec{A}_2$:
\begin{equation}
\boxed{\;
\vec{H}_{\text{sctd}}(\vec{r})=\frac{e^{ik_0 r}}{r}\ \vec{F}_2(\hat{s}), \qquad \text{where} \quad \vec{F}_2(\hat{s})=-\hat{s}\times\Big[\hat{s}\times\Big(\vec{P}'(\hat{s})+\vec{P}''(\hat{s})\Big)\Big].
\;}
\end{equation}
The vector amplitudes are given by:
\begin{equation}
\boxed{\;
\begin{aligned}
\vec{P}'(\hat{s}) &= \frac{k_0^2}{4\pi}\int_V e^{-ik_0\,\hat{s}\cdot\vec{r}'}\,[\varepsilon(\vec{r}')-1]\ \vec{H}(\vec{r}')\,dV',\\[4pt]
\vec{P}''(\hat{s}) &= -\frac{ik_0}{4\pi}\int_V e^{-ik_0\,\hat{s}\cdot\vec{r}'}\ \nabla'\varepsilon(\vec{r}')\times\vec{E}_{int}(\vec{r}')\,dV'.
\end{aligned}
\;}
\end{equation}

\subsection{Physical Interpretation of the Contributions}


The two terms in the scattering amplitude, $\vec{P}'$ and $\vec{P}''$, have distinct physical origins.
\begin{itemize}
    \item $\mathbf{\vec{P}'(\hat{s})}$ represents the radiation from the bulk of the material, driven by the contrast current proportional to $(\varepsilon-1)\vec{H}$. This is analogous to the source term in the E-field equation.
    \item $\mathbf{\vec{P}''(\hat{s})}$ is a unique feature of the MFIE. It represents radiation generated only where the material properties change, i.e., where $\nabla\varepsilon \neq 0$. This term is sourced by the term $\nabla\varepsilon \times \vec{E}$, which can be thought of as an effective magnetic surface current at interfaces.
\end{itemize}
This second term is crucial because it accounts for polarisation effects that are not present in the simpler E-field formulation. The cross product with $\nabla\varepsilon$ can rotate the polarisation of the internal E-field, leading to a more complex polarisation pattern in the scattered magnetic field. For a homogeneous object, this term becomes a surface integral over the object's boundary.


