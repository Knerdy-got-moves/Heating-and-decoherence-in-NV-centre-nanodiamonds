\chapter{Power absorption for beyond Rayleigh size particle: Average Poynting vector}
\begin{comment}
\section{Interference Poynting vector for the $\perp$ channel}

\paragraph{Given fields.}
Let
\[
\Phi:=k_0 R\sin\theta\cos\phi,\qquad \cos\psi=\sin\theta\cos\phi,\qquad
\sin\psi=\sqrt{1-\sin^2\theta\cos^2\phi}.
\]
Define the unit polarization (no radial component)
\[
\hat{\bm D}:=\frac{\sin\phi}{\sin\psi}\,\hat{\bm\theta}
+\frac{\cos\theta\cos\phi}{\sin\psi}\,\hat{\bm\phi}.
\]
Then the \emph{incident} and \emph{scattered} fields in the $\perp$ channel read
\[
\bm E_{\rm inc}=e^{i\Phi}\hat{\bm D},\qquad
\bm H_{\rm inc}=\hat x\times\bm E_{\rm inc},
\]
\[
\bm E_{\rm sctd}=\frac{e^{ik_0 R}}{R}\,F(\psi)\,\hat{\bm D},\qquad
\bm H_{\rm sctd}=\hat{\bm r}\times\bm E_{\rm sctd}.
\]
(The full expressions you provided reduce to these compact forms because
$\bm E_{\rm inc}$ and $\bm E_{\rm sctd}$ are \emph{parallel} and purely transverse.)

\paragraph{Useful identities.}
\begin{align}
\text{(i)}\quad & \hat{\bm r}\cdot\hat{\bm D}=0,\qquad
\hat x = (\sin\theta\cos\phi)\,\hat{\bm r}+(\cos\theta\cos\phi)\,\hat{\bm\theta}-(\sin\phi)\,\hat{\bm\phi},\\
\text{(ii)}\quad & \hat{\bm D}\cdot\hat{\bm D}=
\frac{\sin^2\phi+\cos^2\theta\cos^2\phi}{\sin^2\psi}
=\frac{1-\sin^2\theta\cos^2\phi}{\sin^2\psi}=1,\\
\text{(iii)}\quad &
\hat x\cdot\hat{\bm D}=\frac{\cos\theta\cos\phi\,\sin\phi-\sin\phi\,\cos\theta\cos\phi}{\sin\psi}=0.
\end{align}

\paragraph{Computing $\bm E_{\rm inc}\times\bm H_{\rm sctd}^*$.}
Use $\,\bm H_{\rm sctd}^*=\hat{\bm r}\times\bm E_{\rm sctd}^*\,$ and the triple-product identity
$\,\bm A\times(\hat{\bm r}\times\bm B)=(\bm A\!\cdot\!\bm B)\,\hat{\bm r}-(\bm A\!\cdot\!\hat{\bm r})\,\bm B$.
Since $\bm E_{\rm inc}\perp\hat{\bm r}$ and $\bm E_{\rm sctd}\perp\hat{\bm r}$,
\[
\bm E_{\rm inc}\times\bm H_{\rm sctd}^*
=(\bm E_{\rm inc}\!\cdot\!\bm E_{\rm sctd}^*)\,\hat{\bm r}.
\]
But $\bm E_{\rm inc}\!\cdot\!\bm E_{\rm sctd}^*
=e^{i\Phi}\,\frac{e^{-ik_0 R}}{R}\,F(\psi)^*\,(\hat{\bm D}\!\cdot\!\hat{\bm D})
=\frac{e^{i(\Phi-k_0 R)}}{R}\,F(\psi)^*$ by (ii). Hence
\[
\boxed{\ \bm E_{\rm inc}\times\bm H_{\rm sctd}^*
=\frac{e^{i(\Phi-k_0 R)}}{R}\,F(\psi)^*\,\hat{\bm r}\ }.
\]

\paragraph{Computing $\bm E_{\rm sctd}\times\bm H_{\rm inc}^*$.}
Use $\bm H_{\rm inc}^*=\hat x\times\bm E_{\rm inc}^*$ and
$\bm A\times(\hat x\times\bm B)=(\bm A\!\cdot\!\bm B)\,\hat x-(\bm A\!\cdot\!\hat x)\,\bm B$:
\[
\bm E_{\rm sctd}\times\bm H_{\rm inc}^*
=(\bm E_{\rm sctd}\!\cdot\!\bm E_{\rm inc}^*)\,\hat x
-(\bm E_{\rm sctd}\!\cdot\!\hat x)\,\bm E_{\rm inc}^*.
\]
By (iii) the second term vanishes. The first term equals
\[
\bm E_{\rm sctd}\!\cdot\!\bm E_{\rm inc}^*
=\frac{e^{ik_0 R}}{R}\,F(\psi)\,e^{-i\Phi}\,(\hat{\bm D}\!\cdot\!\hat{\bm D})
=\frac{e^{i(k_0 R-\Phi)}}{R}\,F(\psi).
\]
Therefore
\[
\boxed{\ \bm E_{\rm sctd}\times\bm H_{\rm inc}^*
=\frac{e^{i(k_0 R-\Phi)}}{R}\,F(\psi)\,\hat x\ }.
\]

\paragraph{Assembling the time-averaged interference Poynting vector.}
By definition,
\[
\big\langle\bm S_{\rm int}\big\rangle
=\frac{1}{2c}\Re\!\left(\bm E_{\rm inc}\times\bm H_{\rm sctd}^*
+\bm E_{\rm sctd}\times\bm H_{\rm inc}^*\right)
=\frac{1}{2Rc}\Re\!\left(e^{i(\Phi-k_0 R)}F^*\,\hat{\bm r}
+e^{i(k_0 R-\Phi)}F\,\hat x\right).
\]
Let $F(\psi)=|F(\psi)|\,e^{i\delta(\psi)}$. Then both real parts coincide:
\[
\Re\!\big(e^{i(\Phi-k_0 R)}F^*\big)
=\Re\!\big(e^{i(k_0 R-\Phi)}F\big)
=|F(\psi)|\cos\!\big(k_0 R-\Phi+\delta(\psi)\big).
\]
Hence the compact vector form
\[
\boxed{\;
\big\langle\bm S_{\rm int}\big\rangle
=\frac{\Re\!\big(e^{i(k_0 R-\Phi)}F\big)}{2Rc}\,
\big(\hat{\bm r}+\hat x\big).
\;}
\]

\paragraph{Equivalence of the spherical-component form.}
Using $\hat x=\sin\theta\cos\phi\,\hat{\bm r}
+\cos\theta\cos\phi\,\hat{\bm\theta}-\sin\phi\,\hat{\bm\phi}$,
\[
\boxed{\;
\big\langle\bm S_{\rm int}\big\rangle
=\frac{\Re\!\big(e^{i(k_0 R-\Phi)}F\big)}{2Rc}\,
\Big[\,(1+\sin\theta\cos\phi)\,\hat{\bm r}
+\cos\theta\cos\phi\,\hat{\bm\theta}
-\sin\phi\,\hat{\bm\phi}\,\Big].
\;}
\]

\paragraph{Notes.}
The limit $\sin\psi\to 0$ (forward/back scattering) is taken by continuity; the
unit vector $\hat{\bm D}$ remains well behaved.

\section{Extinction Cross section}

Starting from
\[
\big\langle\bm S_{\rm int}\big\rangle\cdot\hat{\bm r}
=\frac{\Re\!\big(e^{i(k_0 R-\Phi)}F\big)}{2Rc}\,
\big(1+\sin\theta\cos\phi\big),
\quad \cos\psi=\sin\theta\cos\phi,
\]
and integrating over a sphere of radius \(R\):
\[
\oint_{S_R}\!\big\langle\bm S_{\rm int}\big\rangle\cdot\hat{\bm r}\,dA
=\frac{R}{2c}\!\int_0^{2\pi}\!\!\int_0^\pi
\Re\!\big(e^{i(k_0 R-k_0 R\sin\theta\cos\phi)}F(sin\theta\cos\phi)\big)
\,(1+\sin\theta\cos\phi)\,\sin\theta\,d\theta\,d\phi.
\]

\paragraph{Lemma (axis reduction).}
If an integrand on the unit sphere depends only on \(x:=\sin\theta\cos\phi\) (i.e. the
projection onto the \(x\)-axis), then
\[
\int_0^{2\pi}\!\!\int_0^\pi g(\sin\theta\cos\phi)\,\sin\theta\,d\theta\,d\phi
=2\pi\int_{-1}^{1} g(x)\,dx.
\]
\textbf{Proof.}
A point on the unit sphere can be written as
\[
(x,y,z)=(\sin\theta\cos\phi,\ \sin\theta\sin\phi,\ \cos\theta),
\quad \theta\in[0,\pi],\ \phi\in[0,2\pi],
\]
so \(x=\sin\theta\cos\phi\).
Reparameterize the sphere using spherical angles with the \(x\)-axis as the polar axis:
\[
(x,y,z)=(\cos\Theta,\ \sin\Theta\cos\Phi,\ \sin\Theta\sin\Phi),
\quad \Theta\in[0,\pi],\ \Phi\in[0,2\pi].
\]
This change of coordinates is a rotation, so the surface element is unchanged:
\[
\sin\theta\,d\theta\,d\phi=\sin\Theta\,d\Theta\,d\Phi.
\]
Since \(x=\cos\Theta\) in these coordinates,
\begin{align*}
\int_0^{2\pi}\!\!\int_0^\pi g(\sin\theta\cos\phi)\,\sin\theta\,d\theta\,d\phi
&=\int_0^{2\pi}\!\!\int_0^\pi g(\cos\Theta)\,\sin\Theta\,d\Theta\,d\Phi \\
&=2\pi \int_0^\pi g(\cos\Theta)\,\sin\Theta\,d\Theta \\
&\overset{u=\cos\Theta}{=} 2\pi \int_{1}^{-1} g(u)\,(-du)
=2\pi \int_{-1}^{1} g(u)\,du.
\end{align*}
\qed

Here the integrand depends on \(x=\sin\theta\cos\phi\) only;
With \(x=\cos\psi\) and \(F(\psi)=F(\arccos x)=:F(x)\), the flux becomes
\[
\boxed{\;
\oint_{S_R}\!\big\langle\bm S_{\rm int}\big\rangle\cdot\hat{\bm r}\,dA
=\frac{\pi R}{c}\,\Re\!\left[
e^{i k_0 R}\int_{-1}^{1} (1+x)\,F(x)\,e^{-i k_0 R x}\,dx\right].
\;}
\]
This is an \emph{exact} 1D integral valid for all \(k_0R\).

\paragraph{Exact form of \(F(X)\).}
\[
X=k_0 a\sqrt{1+m^2-2mx},\qquad
F(x)=K\!\left[\frac{3j_1(X)}{X}+\gamma(X)\right],\qquad
K=k_0^2\frac{(m^2-1)a^3}{3}.
\]
Solving for \(x\) in terms of \(X\):
\[
x(X)=\frac{1+m^2-(X/k_0a)^2}{2m},\qquad
dx=\frac{-X}{m(k_0a)^2}\,dX.
\]
The endpoints are the branch–continuous images of \(x=\pm1\):
\[
X_-:=k_0a\,(m-1),\qquad X_+:=k_0a\,(m+1)
\quad\text{(choose a continuous branch for the square root)}.
\]
Also \(1+x(X)=\dfrac{(m+1)^2-(X/k_0a)^2}{2m}\) and
\(e^{-i k_0 R x(X)}=e^{-\frac{i k_0 R}{2m}(1+m^2)}\,
e^{\,\frac{i k_0 R}{2m}\,(X/k_0a)^2}\).
Thus
\[
\boxed{\;
\begin{aligned}
\oint_{S_R}\!\big\langle\bm S_{\rm int}\big\rangle\cdot\hat{\bm r}\,dA
&=\frac{\pi R}{c}\,\Re\!\Bigg[
e^{-\frac{i k_0 R}{2m}(m-1)^2}
\int_{X_-}^{X_+}
\frac{\big((m+1)^2-(X/k_0a)^2\big)\,X}{2m^2(k_0a)^2}\,
F(X)\,e^{\,i\alpha X^2}\,dX
\Bigg],\\[4pt]
\alpha&:=\frac{k_0 R}{2m\,(k_0a)^2}=\frac{R}{2m\,k_0 a^2}.
\end{aligned}
\;}
\]

\subsection{Evaluation of the \texorpdfstring{$\gamma$}{gamma}-term for $\gamma(X)=X^{-3/2}$}

We work with the sphere $S_a$ and use the exact 1D reduction obtained earlier. 
With
\[
F(X)=K\!\left[\frac{3\,j_1(X)}{X}+\gamma(X)\right],\quad 
\gamma(X)=X^{-3/2},\quad 
K=k_0^2\frac{(m^2-1)a^3}{3},
\]
define
\[
X_\pm := k_0 a\,(m\pm 1),\qquad 
u_\pm := X_\pm^2=(k_0 a)^2(m\pm 1)^2,\qquad
\alpha := \frac{1}{2m\,k_0 a}.
\]
The contribution of the $\gamma$-piece to the interference flux is
\[
\frac{\pi a}{c}\,\Re\!\left\{
e^{-\,\frac{i k_0 a}{2m}(m-1)^2}\; I_\gamma
\right\},
\]
where
\[
I_\gamma=\frac{K}{2m^2(k_0a)^2}\int_{X_-}^{X_+}
\Big[(m+1)^2\,X^{-1/2}-\frac{1}{(k_0a)^2}\,X^{3/2}\Big]\,
e^{\,i\alpha X^2}\,dX.
\]

Make the substitution $u=X^2$ (so $dX=\tfrac{du}{2X}$). Then
\[
\int X^{-\tfrac12} e^{\,i\alpha X^2}\,dX
=\frac12\!\int u^{-\tfrac34} e^{\,i\alpha u}\,du
=\frac12\,(-i\alpha)^{-1/4}\,\Gamma\!\left(\frac14,\,-i\alpha u\right),
\]
\[
\int X^{\tfrac32} e^{\,i\alpha X^2}\,dX
=\frac12\!\int u^{\tfrac14} e^{\,i\alpha u}\,du
=\frac12\,(-i\alpha)^{-5/4}\,\Gamma\!\left(\frac54,\,-i\alpha u\right),
\]
where $\Gamma(\nu,z)$ is the upper incomplete Gamma function. Evaluating between 
$u_-$ and $u_+$ gives
\[
I_\gamma=\frac{K}{4m^2(k_0a)^2}\left[
(m+1)^2(-i\alpha)^{-1/4}\,\Delta\Gamma_{1/4}
-\frac{1}{(k_0a)^2}(-i\alpha)^{-5/4}\,\Delta\Gamma_{5/4}
\right],
\]
with the shorthand
\[
\Delta\Gamma_{\nu}
:=\Gamma\!\left(\nu,\,-i\alpha u_+\right)-\Gamma\!\left(\nu,\,-i\alpha u_-\right)
=\Gamma\!\left(\nu,\,-\,\frac{i k_0 a}{2m}(m+1)^2\right)
-\Gamma\!\left(\nu,\,-\,\frac{i k_0 a}{2m}(m-1)^2\right).
\]

Therefore the $\gamma$-contribution to the surface integral is
\[
\boxed{\
\frac{\pi a}{c}\,\Re\!\left\{
e^{-\,\frac{i k_0 a}{2m}(m-1)^2}
\cdot
\frac{K}{4m^2(k_0a)^2}\left[
(m+1)^2(-i\alpha)^{-1/4}\,\Delta\Gamma_{1/4}
-\frac{1}{(k_0a)^2}(-i\alpha)^{-5/4}\,\Delta\Gamma_{5/4}
\right]
\right\}.
}
\]
All branch choices are understood as principal branches that continue continuously between the endpoints $X_\pm$.

%--- Step C: Add the (3 j1(X)/X) contribution and assemble the full result -----

\subsection{The $\frac{3\,j_1(X)}{X}$ contribution and the full surface integral}

Recall the exact $X$-integral representation on $S_a$:
\[
\oint_{S_a}\!\langle\bm S_{\rm int}\rangle\!\cdot\!\hat{\bm r}\,dA
=\pi a\,\Re\!\left\{
e^{-\,\frac{i k_0 a}{2m}(m-1)^2}\,
\int_{X_-}^{X_+}
\frac{\big((m+1)^2-(X/k_0a)^2\big)\,X}{2m^2(k_0a)^2}\,
F(X)\,e^{\,i\alpha X^2}\,dX
\right\},
\]
where
\[
F(X)=K\!\left[\frac{3\,j_1(X)}{X}+\gamma(X)\right],\qquad
\gamma(X)=X^{-3/2},\qquad
K=k_0^2\frac{(m^2-1)a^3}{3},\qquad
\alpha=\frac{1}{2m\,k_0 a},
\]
and the endpoints
\[
X_\pm:=k_0 a\,(m\pm 1),\qquad u_\pm:=X_\pm^2=(k_0 a)^2(m\pm 1)^2.
\]

\paragraph{The $3j_1(X)/X$ piece.}
Define
\[
I_{j_1}
:=\frac{3K}{2m^2(k_0a)^2}\!\int_{X_-}^{X_+}\!
\Big[(m+1)^2-\frac{X^2}{(k_0a)^2}\Big]\,j_1(X)\,e^{\,i\alpha X^2}\,dX.
\]
Introduce
\[
J(\alpha):=\int_{X_-}^{X_+} j_1(X)\,e^{\,i\alpha X^2}\,dX.
\]
Using \(j_1(X)=-\dfrac{d}{dX}\!\left(\dfrac{\sin X}{X}\right)\) and integrating by parts,
\[
J(\alpha)= -\Big[\frac{\sin X}{X}e^{\,i\alpha X^2}\Big]_{X_-}^{X_+}
+2i\alpha\int_{X_-}^{X_+}\!\sin X\,e^{\,i\alpha X^2}\,dX.
\]
Write the last integral via a completed square:
\[
\int_{X_-}^{X_+}\!\sin X\,e^{\,i\alpha X^2}\,dX
=\Im\!\left\{\,\int_{X_-}^{X_+}\!e^{\,i\alpha X^2+iX}\,dX\right\}
=\Im\!\left\{\,e^{-\frac{i}{4\alpha}}\,
\frac{\sqrt{\pi}\,e^{i\pi/4}}{2\sqrt{\alpha}}\,
\Big[\operatorname{erf}\big(z_+\big)-\operatorname{erf}\big(z_-\big)\Big]\right\},
\]
where
\[
z_\pm:=e^{i\pi/4}\sqrt{\alpha}\left(X_\pm+\frac{1}{2\alpha}\right).
\]
Hence
\[
\boxed{\
J(\alpha)= -\Big[\frac{\sin X}{X}e^{\,i\alpha X^2}\Big]_{X_-}^{X_+}
+2i\alpha\;\Im\!\left\{\,e^{-\frac{i}{4\alpha}}\,
\frac{\sqrt{\pi}\,e^{i\pi/4}}{2\sqrt{\alpha}}\,
\Big[\operatorname{erf}\big(z_+\big)-\operatorname{erf}\big(z_-\big)\Big]\right\}.
}
\]

To handle the $X^2 j_1(X)$ term, use
\[
\frac{\partial}{\partial\alpha}\Big(e^{\,i\alpha X^2}\Big)=iX^2 e^{\,i\alpha X^2}
\quad\Rightarrow\quad
\int_{X_-}^{X_+}\!X^2 j_1(X)\,e^{\,i\alpha X^2}\,dX
=\frac{1}{i}\,\frac{d}{d\alpha}J(\alpha).
\]
Therefore,
\[
\boxed{\
I_{j_1}
=\frac{3K}{2m^2(k_0a)^2}\left[
(m+1)^2\,J(\alpha)
-\frac{1}{(k_0a)^2}\,\frac{1}{i}\,\frac{dJ}{d\alpha}
\right].
}
\]
The derivative $dJ/d\alpha$ follows by differentiating the boxed $J(\alpha)$ using
$\dfrac{d}{d\alpha}\!\left(\alpha^{-1/2}e^{-i/(4\alpha)}\right)
=\alpha^{-1/2}e^{-i/(4\alpha)}\!\left(\frac{i}{4\alpha^2}-\frac{1}{2\alpha}\right)$
and $\dfrac{d}{d\alpha}\operatorname{erf}(z)=\dfrac{2}{\sqrt{\pi}}e^{-z^2}\,z'(\alpha)$ with
\[
z'_\pm(\alpha)
=e^{i\pi/4}\left[
\frac{X_\pm+\frac{1}{2\alpha}}{2\sqrt{\alpha}}
-\frac{\sqrt{\alpha}}{2\alpha^2}
\right].
\]

\paragraph{The $\gamma$-piece.}
With $\gamma(X)=X^{-3/2}$ we already obtained
\[
I_\gamma=\frac{K}{4m^2(k_0a)^2}\left[
(m+1)^2(-i\alpha)^{-1/4}\,\Delta\Gamma_{1/4}
-\frac{1}{(k_0a)^2}(-i\alpha)^{-5/4}\,\Delta\Gamma_{5/4}
\right],
\]
where
\[
\Delta\Gamma_{\nu}:=\Gamma\!\left(\nu,\,-i\alpha u_+\right)-\Gamma\!\left(\nu,\,-i\alpha u_-\right),
\qquad u_\pm=(k_0 a)^2(m\pm 1)^2.
\]

\paragraph{Final assembled result.}
Combining $I_{j_1}$ and $I_\gamma$,
\[
\boxed{\
\begin{aligned}
\sigma_{ext}
&=\pi a\,
\Re\!\left\{
e^{-\,\frac{i k_0 a}{2m}(m-1)^2}
\left(I_{j_1}+I_\gamma\right)
\right\},\\[4pt]
I_{j_1}
&=\frac{3K}{2m^2(k_0a)^2}\left[
(m+1)^2\,J(\alpha)
-\frac{1}{(k_0a)^2}\,\frac{1}{i}\,\frac{dJ}{d\alpha}
\right],\\[4pt]
J(\alpha)
&= -\Big[\frac{\sin X}{X}e^{\,i\alpha X^2}\Big]_{X_-}^{X_+}
+2i\alpha\;\Im\!\left\{\,e^{-\frac{i}{4\alpha}}\,
\frac{\sqrt{\pi}\,e^{i\pi/4}}{2\sqrt{\alpha}}\,
\Big[\operatorname{erf}\big(z_+\big)-\operatorname{erf}\big(z_-\big)\Big]\right\},\\[4pt]
z_\pm&=e^{i\pi/4}\sqrt{\alpha}\left(X_\pm+\frac{1}{2\alpha}\right),\\[4pt]
I_\gamma
&=\frac{K}{4m^2(k_0a)^2}\left[
(m+1)^2(-i\alpha)^{-1/4}\,\Delta\Gamma_{1/4}
-\frac{1}{(k_0a)^2}(-i\alpha)^{-5/4}\,\Delta\Gamma_{5/4}
\right],\\[4pt]
\Delta\Gamma_{\nu}
&=\Gamma\!\left(\nu,\,-i\alpha (k_0 a)^2(m+1)^2\right)
-\Gamma\!\left(\nu,\,-i\alpha (k_0 a)^2(m-1)^2\right),\\[4pt]
K&=k_0^2\frac{(m^2-1)a^3}{3},\qquad \alpha=\frac{1}{2m\,k_0 a},\qquad X_\pm=k_0 a\,(m\pm 1).
\end{aligned}
}
\]

\noindent
This expression is exact (no $k_0 a$ or $k_0 R$ asymptotics). The only
special functions are the error function $\operatorname{erf}$ and the upper incomplete gamma
$\Gamma(\nu,z)$, plus the elementary derivative $dJ/d\alpha$, which you can evaluate
symbolically or numerically from the given $J(\alpha)$.
\begin{comment}
\section{Interference flux Leading order in the particle size $a$}

We use the exact result
\[
\oint_{S_a}\!\langle\bm S_{\rm int}\rangle\!\cdot\!\hat{\bm r}\,dA
=\frac{\pi a}{c}\,
\Re\!\left\{
e^{-\,\frac{i k_0 a}{2m}(m-1)^2}
\left(I_{j_1}+I_\gamma\right)
\right\},
\]
with
\[
I_{j_1}
=\frac{3K}{2m^2(k_0a)^2}\left[
(m+1)^2\,J(\alpha)
-\frac{1}{(k_0a)^2}\,\frac{1}{i}\,\frac{dJ}{d\alpha}
\right],\qquad
K=k_0^2\frac{(m^2-1)a^3}{3},\qquad
\alpha=\frac{1}{2m\,k_0 a},
\]
and
\[
J(\alpha)
= -\Big[\frac{\sin X}{X}e^{\,i\alpha X^2}\Big]_{X_-}^{X_+}
+2i\alpha\;\Im\!\left\{\,e^{-\frac{i}{4\alpha}}\,
\frac{\sqrt{\pi}\,e^{i\pi/4}}{2\sqrt{\alpha}}\,
\Big[\operatorname{erf}\big(z_+\big)-\operatorname{erf}\big(z_-\big)\Big]\right\}.
\]

\paragraph{Small-$a$ scalings.}
As $a\to 0$,
\[
X_\pm=k_0 a\,(m\pm 1)=\mathcal{O}(a),\qquad
\alpha=\frac{1}{2m\,k_0 a}=\mathcal{O}\!\big(a^{-1}\big).
\]
Expanding the $J$-pieces at leading order:
\[
\Big[\tfrac{\sin X}{X}e^{i\alpha X^2}\Big]_{X_-}^{X_+}
=1-\frac{X_+^2-X_-^2}{6}+\mathcal{O}(a^3)
=1-\frac{4m(k_0a)^2}{6}+\mathcal{O}(a^3),
\]
hence its contribution to $J(\alpha)$ is $\mathcal{O}(a^2)$.
For the Fresnel error-function piece one finds (using
$z_\pm=e^{i\pi/4}\sqrt{\alpha}(X_\pm+\tfrac{1}{2\alpha})$ and
$\operatorname{erf}z= \tfrac{2}{\sqrt{\pi}}z+\mathcal{O}(z^3)$)
\[
\Im\!\left\{\,e^{-\frac{i}{4\alpha}}\,
\frac{\sqrt{\pi}e^{i\pi/4}}{2\sqrt{\alpha}}\,
\big[\operatorname{erf}(z_+)-\operatorname{erf}(z_-)\big]\right\}
=2k_0 a\,\cos\!\Big(\tfrac{1}{4\alpha}\Big)+\mathcal{O}(a^2),
\]
so
\[
J(\alpha)=2i\alpha\cdot 2k_0 a\,\cos\!\Big(\tfrac{1}{4\alpha}\Big)
+\mathcal{O}(a^2)=\frac{2i}{m}\,\cos\!\Big(\tfrac{1}{4\alpha}\Big)
+\mathcal{O}(a^2),
\]
and therefore
\[
\frac{dJ}{d\alpha}
=4i k_0 a\,\cos\!\Big(\tfrac{1}{4\alpha}\Big)+\mathcal{O}(a^2).
\]

\paragraph{Dominant piece in $I_{j_1}$.}
Insert the above into $I_{j_1}$. The bracket contains
\[
(m+1)^2 J(\alpha)=\mathcal{O}(1),\qquad
-\frac{1}{(k_0a)^2}\,\frac{1}{i}\frac{dJ}{d\alpha}
= -\,\frac{4}{k_0 a}\,\cos\!\Big(\tfrac{1}{4\alpha}\Big)
+ \mathcal{O}(1).
\]
The second term dominates as $a\to 0$. Using $K=k_0^2(m^2-1)a^3/3$,
\[
I_{j_1}=
\frac{3K}{2m^2(k_0a)^2}\left[-\,\frac{4}{k_0 a}\,\cos\!\Big(\tfrac{1}{4\alpha}\Big)\right]
+\mathcal{O}(a)
= -\,\frac{2(m^2-1)}{m^2 k_0}\,\cos\!\Big(\tfrac{1}{4\alpha}\Big)
+\mathcal{O}(a).
\]

\paragraph{Subleading $\gamma$-piece.}
From Step B (with $\gamma(X)=X^{-3/2}$) one finds
$I_\gamma=\mathcal{O}\!\big(a^{3/2}\big)$; hence
$\frac{\pi a}{c}\,I_\gamma=\mathcal{O}\!\big(a^{5/2}\big)$ and is negligible at leading order.

\paragraph{Assembling the leading term.}
Thus
\[
\oint_{S_a}\!\langle\bm S_{\rm int}\rangle\!\cdot\!\hat{\bm r}\,dA
= \frac{\pi a}{c}\,
\Re\!\left\{
e^{-\,\frac{i k_0 a}{2m}(m-1)^2}
\left[-\,\frac{2(m^2-1)}{m^2 k_0}\,\cos\!\Big(\tfrac{1}{4\alpha}\Big)\right]
\right\}
+\mathcal{O}(a^2).
\]
Since $\tfrac{1}{4\alpha}=\tfrac{k_0 a}{2m}$, the two cosine phases coincide to
$\mathcal{O}(a)$, and we may keep the compact leading form
\[
\boxed{\
\oint_{S_a}\!\langle\bm S_{\rm int}\rangle\!\cdot\!\hat{\bm r}\,dA
\;=\; -\,\frac{2\pi a}{c\,k_0}\,
\Re\!\left[
\frac{m^2-1}{m^2}\,
\exp\!\left(-\,\frac{i k_0 a}{2m}(m-1)^2\right)
\right]
\;+\;\mathcal{O}(a^2).
}
\]

\paragraph{Very-small-$a$ simplification.}
Expanding the exponential as $a\to 0$,
\[
\exp\!\left(-\,\frac{i k_0 a}{2m}(m-1)^2\right)=1+\mathcal{O}(a),
\]
so the leading $a$-scaling is
\[
\boxed{\
\oint_{S_a}\!\langle\bm S_{\rm int}\rangle\!\cdot\!\hat{\bm r}\,dA
\;=\; -\,\frac{2\pi a}{c\,k_0}\,
\Re\!\left(\frac{m^2-1}{m^2}\right)
\;+\;\mathcal{O}(a^2).
}
\]
This shows the net interference flux scales linearly with $a$ at small size,
with subleading corrections of order $a^2$ (and an even smaller
$\mathcal{O}(a^{5/2})$ contribution from the $\gamma$-term).
\end{comment}




\begin{comment}

\section{Scattered Poynting vector \texorpdfstring{$\langle \bm S_{\rm sca}\rangle$}{<S_sca>}}

We are given (Gaussian units, $\mu=1$, $e^{-i\omega t}$)
\[
\bm E_{\rm sca}
=\frac{e^{ik_0 R}}{R}\,F(\psi)\,
\underbrace{\left(\frac{\sin\phi}{\sin\psi}\,\hat{\bm\theta}
+\frac{\cos\theta\cos\phi}{\sin\psi}\,\hat{\bm\phi}\right)}_{:=\ \hat{\bm D}},
\qquad
\bm H_{\rm sca}=\hat{\bm r}\times\bm E_{\rm sca}.
\]
Note that $\hat{\bm D}\!\cdot\!\hat{\bm r}=0$ and
\[
|\hat{\bm D}|^2
=\frac{\sin^2\phi+\cos^2\theta\cos^2\phi}{\sin^2\psi}
=\frac{1-\sin^2\theta\cos^2\phi}{\sin^2\psi}
=\frac{\sin^2\psi}{\sin^2\psi}=1,
\]
so $\hat{\bm D}$ is a real unit vector.

Time-averaged scattered Poynting vector:
\[
\langle \bm S_{\rm sca}\rangle
=\frac{1}{2c}\,\Re\!\big(\bm E_{\rm sca}\times \bm H_{\rm sca}^*\big)
=\frac{1}{2c}\,\Re\!\big(\bm E_{\rm sca}\times (\hat{\bm r}\times \bm E_{\rm sca}^*)\big).
\]
Use the triple-product identity:
\[
\bm E_{\rm sca}\times (\hat{\bm r}\times \bm E_{\rm sca}^*)
=(\bm E_{\rm sca}\!\cdot\!\bm E_{\rm sca}^*)\,\hat{\bm r}
-(\bm E_{\rm sca}\!\cdot\!\hat{\bm r})\,\bm E_{\rm sca}^*.
\]
Since $\bm E_{\rm sca}\perp\hat{\bm r}$, the second term vanishes. With
$\bm E_{\rm sca}=\dfrac{e^{ik_0 R}}{R}F(\psi)\,\hat{\bm D}$ and $|\hat{\bm D}|=1$,
\[
\bm E_{\rm sca}\!\cdot\!\bm E_{\rm sca}^*=\frac{|F(\psi)|^2}{R^2}.
\]
Therefore
\[
\boxed{\
\langle \bm S_{\rm sca}\rangle
=\frac{1}{2c}\,\frac{|F(\psi)|^2}{R^2}\;\hat{\bm r}.\
}
\]
Equivalently, the radial flux density is
\[
\boxed{\
\langle \bm S_{\rm sca}\rangle\!\cdot\!\hat{\bm r}
=\frac{1}{2c}\,\frac{|F(\psi)|^2}{R^2}.\
}
\]
\section{Scattering cross section}

From
\[
\langle \bm S_{\rm sca}\rangle
=\frac{1}{2c}\,\frac{|F(\psi)|^2}{R^2}\,\hat{\bm r}
\quad\Rightarrow\quad
\langle \bm S_{\rm sca}\rangle\!\cdot\!\hat{\bm r}
=\frac{1}{2c}\,\frac{|F(\psi)|^2}{R^2},
\]
the scattered power through a sphere of radius $R$ is
\[
P_{\rm sca}=\oint_{S_R}\langle \bm S_{\rm sca}\rangle\!\cdot\!\hat{\bm r}\,dA
=\frac{1}{2c}\int_{4\pi}|F(\psi)|^2\,d\Omega,
\qquad dA=R^2\,d\Omega.
\]

\paragraph{Axis reduction (exact).}
Let $x:=\sin\theta\cos\phi=\cos\psi$. Since $|F|^2$ depends only on $x$,
\[
\int_0^{2\pi}\!\!\int_0^\pi g(\sin\theta\cos\phi)\,\sin\theta\,d\theta\,d\phi
=2\pi\int_{-1}^{1} g(x)\,dx.
\]
Hence
\[
\boxed{\
P_{\rm sca}
=\frac{\pi}{c}\int_{-1}^{1}\!|F(x)|^2\,dx,
\qquad x=\cos\psi=\sin\theta\cos\phi.\
}
\]

\paragraph{Differential and total scattering cross-sections.}
With incident intensity $\langle S_{\rm inc}\rangle=\tfrac{1}{2c}$ (unit
incident field amplitude),
\[
\frac{d\sigma_{\rm sca}}{d\Omega}
=\frac{R^2\langle \bm S_{\rm sca}\rangle\!\cdot\!\hat{\bm r}}{\langle S_{\rm inc}\rangle}
=|F(\psi)|^2,
\qquad
\boxed{\ \sigma_{\rm sca}=\int_{4\pi}\!|F(\psi)|^2\,d\Omega
=2\pi\int_{-1}^{1}\!|F(x)|^2\,dx.\ }
\]

\paragraph{Change of variable to your $X$ (Mie-like) argument (exact).}
Recall
\[
X=k_0 a\sqrt{1+m^2-2mx},\quad
x(X)=\frac{1+m^2-(X/k_0a)^2}{2m},\quad
dx=\frac{-X}{m(k_0a)^2}\,dX,
\]
with endpoints
\[
X_-:=k_0a\,(m-1),\qquad X_+:=k_0a\,(m+1).
\]
Then
\[
\boxed{\
\sigma_{\rm sca}
=\frac{2\pi}{m(k_0a)^2}\int_{X_-}^{X_+}\!X\,|F(X)|^2\,dX.
\ }
\]
Here, $F$ may be taken as
\[
F(X)=K\!\left[\frac{3\,j_1(X)}{X}+\gamma(X)\right],
\qquad K=k_0^2\frac{(m^2-1)a^3}{3},
\]
with your chosen $\gamma(X)$.
\subsection{Total scattering cross-section with $\gamma(X)=X^{-3/2}$}

\paragraph{Setup.}
\[
\sigma_{\rm sca}=\int_{4\pi}|F(\psi)|^2\,d\Omega
=2\pi\int_{-1}^{1}\!|F(x)|^2\,dx,
\qquad x=\cos\psi=\sin\theta\cos\phi.
\]
Using \(X=k_0a\sqrt{1+m^2-2mx}\) and \(dx=\dfrac{-X}{m(k_0a)^2}\,dX\) (continuous branch),
\[
\sigma_{\rm sca}
=\frac{2\pi}{m(k_0a)^2}\int_{X_-}^{X_+}\!X\,|F(X)|^2\,dX,
\qquad
X_\pm:=k_0a\,(m\pm 1).
\]

\paragraph{Amplitude and decomposition.}
Take
\[
F(X)=K\!\left[\frac{3\,j_1(X)}{X}+\gamma(X)\right],\qquad
\gamma(X)=X^{-3/2},\qquad
K=\frac{k_0^2(m^2-1)a^3}{3}.
\]
Hence
\[
|F|^2=|K|^2\left[
\left|\frac{3j_1}{X}\right|^2
+2\,\Re\!\left(\frac{3j_1}{X}\,\gamma^*\right)
+|\gamma|^2\right].
\]

\paragraph{Nonabsorbing case (real $m$; $X\in\mathbb{R}_+$) in closed form.}
If $m$ is real (extend by analytic continuation for complex $m$; see remark below),
all quantities under the $X$–integral are real, so
\[
\sigma_{\rm sca}
=\frac{2\pi|K|^2}{m(k_0a)^2}\Big[
9\,\mathcal{I}_1\big|_{X_-}^{X_+}
+6\,\mathcal{I}_2\big|_{X_-}^{X_+}
+\mathcal{I}_3\big|_{X_-}^{X_+}
\Big],
\]
with the primitives
\[
\boxed{\
\begin{aligned}
\mathcal{I}_1(X)
&=\int \frac{j_1(X)^2}{X}\,dX
= -\,\frac{1}{4X^2}
+\frac{\sin X\,\cos X}{2X^3}
-\frac{\sin^2 X}{4X^4},
\\[6pt]
\mathcal{I}_2(X)
&=\int \frac{3\,j_1(X)}{X^{3/2}}\,dX
=3\!\left[
\int \frac{\sin X}{X^{7/2}}\,dX
-\int \frac{\cos X}{X^{5/2}}\,dX
\right],
\\[4pt]
&\hspace{1.5em}=
3\left\{
\Im\!\Big(i^{5/2}\,\Gamma\!\big(-\tfrac{5}{2},-iX\big)\Big)
-\Re\!\Big(i^{3/2}\,\Gamma\!\big(-\tfrac{3}{2},-iX\big)\Big)
\right\},
\\[8pt]
\mathcal{I}_3(X)
&=\int X\cdot X^{-3}\,dX
=\int X^{-2}\,dX
= -\,\frac{1}{X}.
\end{aligned}
}
\]
Here $\Gamma(\nu,z)$ is the upper incomplete Gamma function and
we used the identity
\(\displaystyle \int x^{\nu-1}e^{ix}\,dx=i^{-\nu}\Gamma(\nu,-ix)\)
(analytic continuation understood). Therefore
\[
\boxed{\
\begin{aligned}
\sigma_{\rm sca}
&=\frac{2\pi|K|^2}{m(k_0a)^2}\Bigg[
9\Big(\mathcal{I}_1(X_+)-\mathcal{I}_1(X_-)\Big)
+6\Big(\mathcal{I}_2(X_+)-\mathcal{I}_2(X_-)\Big)\\
&\hspace{10em}
+\Big(\mathcal{I}_3(X_+)-\mathcal{I}_3(X_-)\Big)
\Bigg],
\\[4pt]
X_\pm&=k_0a\,(m\pm 1),\qquad
K=\frac{k_0^2(m^2-1)a^3}{3}.
\end{aligned}
}
\]

\paragraph{Remarks.}
1) For real $m>0$ one has $X_\pm>0$, so the above expressions are elementary
(trig functions and incomplete gamma at real arguments).\\
2) For complex $m$, the same boxed formula holds by analytic continuation:
use the continuous branch from $X_-$ to $X_+$ for all fractional powers and
$\Gamma(\nu,-iX)$; the final $\sigma_{\rm sca}$ is real because it equals
\(\int_{4\pi}|F(\psi)|^2 d\Omega\).\\
3) $\mathcal{I}_2$ can also be written
with generalized cosine/sine integrals, but the incomplete-gamma form is
the most compact.

\section{Absorption cross-section}

By definition,
\[
\sigma_{\rm abs}=\sigma_{\rm ext}-\sigma_{\rm sca}.
\]

From the given exact results,
\[
\sigma_{\rm ext}
=\frac{\pi a}{c}\,
\Re\!\left\{
e^{-\,\frac{i k_0 a}{2m}(m-1)^2}
\left(I_{j_1}+I_\gamma\right)
\right\},
\]
and
\[
\sigma_{\rm sca}
=\frac{2\pi|K|^2}{m(k_0a)^2}\Bigg[
9\Big(\mathcal{I}_1(X_+)-\mathcal{I}_1(X_-)\Big)
+6\Big(\mathcal{I}_2(X_+)-\mathcal{I}_2(X_-)\Big)
+\Big(\mathcal{I}_3(X_+)-\mathcal{I}_3(X_-)\Big)
\Bigg],
\]
with
\[
\begin{aligned}
\mathcal{I}_1(X)
&= -\,\frac{1}{4X^2}
+\frac{\sin X\,\cos X}{2X^3}
-\frac{\sin^2 X}{4X^4},\\[4pt]
\mathcal{I}_2(X)
&=3\left\{
\Im\!\Big(i^{5/2}\,\Gamma\!\big(-\tfrac{5}{2},-iX\big)\Big)
-\Re\!\Big(i^{3/2}\,\Gamma\!\big(-\tfrac{3}{2},-iX\big)\Big)
\right\},\\[4pt]
\mathcal{I}_3(X)
&= -\,\frac{1}{X},
\end{aligned}
\qquad
X_\pm=k_0a\,(m\pm 1),
\]
and
\[
K=\frac{k_0^2(m^2-1)a^3}{3},\qquad
\alpha=\frac{1}{2m\,k_0 a},\qquad
\gamma(X)=X^{-3/2}.
\]
Here \(I_{j_1}\), \(I_\gamma\) are (from the interference integral)
\[
I_{j_1}
=\frac{3K}{2m^2(k_0a)^2}\left[
(m+1)^2\,J(\alpha)
-\frac{1}{(k_0a)^2}\,\frac{1}{i}\,\frac{dJ}{d\alpha}
\right],
\]
\[
J(\alpha)
= -\Big[\frac{\sin X}{X}e^{\,i\alpha X^2}\Big]_{X_-}^{X_+}
+2i\alpha\;\Im\!\left\{\,e^{-\frac{i}{4\alpha}}\,
\frac{\sqrt{\pi}\,e^{i\pi/4}}{2\sqrt{\alpha}}\,
\Big[\operatorname{erf}\big(z_+\big)-\operatorname{erf}\big(z_-\big)\Big]\right\},
\]
\[
z_\pm=e^{i\pi/4}\sqrt{\alpha}\left(X_\pm+\frac{1}{2\alpha}\right),
\]
\[
I_\gamma
=\frac{K}{4m^2(k_0a)^2}\left[
(m+1)^2(-i\alpha)^{-1/4}\,\Delta\Gamma_{1/4}
-\frac{1}{(k_0a)^2}(-i\alpha)^{-5/4}\,\Delta\Gamma_{5/4}
\right],
\]
\[
\Delta\Gamma_{\nu}
=\Gamma\!\left(\nu,\,-i\alpha (k_0 a)^2(m+1)^2\right)
-\Gamma\!\left(\nu,\,-i\alpha (k_0 a)^2(m-1)^2\right).
\]

Therefore the absorption cross-section is
\[
\boxed{\
\begin{aligned}
\sigma_{\rm abs}
&=\pi a\,
\Re\!\left\{
e^{-\,\frac{i k_0 a}{2m}(m-1)^2}
\left(I_{j_1}+I_\gamma\right)
\right\}\\
&\quad -\;
\frac{2\pi|K|^2}{m(k_0a)^2}\Bigg[
9\Big(\mathcal{I}_1(X_+)-\mathcal{I}_1(X_-)\Big)
+6\Big(\mathcal{I}_2(X_+)-\mathcal{I}_2(X_-)\Big)
+\Big(\mathcal{I}_3(X_+)-\mathcal{I}_3(X_-)\Big)
\Bigg].
\end{aligned}
}
\]

Using \(|K|^2=\dfrac{k_0^4 a^6}{9}\,|m^2-1|^2\),
\[
\frac{2\pi|K|^2}{m(k_0a)^2}
=\frac{2\pi\,k_0^2 a^4}{9}\,\frac{|m^2-1|^2}{m},
\]
so the second line of the boxed formula can be written with this prefactor if desired.

\medskip
All expressions are exact for the chosen \(\gamma(X)=X^{-3/2}\); for complex \(m\) use analytic continuation (continuous branches) for the special functions and endpoints \(X_\pm\). These hold in the far-field limit, i.e, $k_0a\gg 1$.
\end{comment}